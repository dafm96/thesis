\chapter{Developed Work}
\label{cha:developed_work}

So far, the developed work has been focused on the player tracking. The extraction of metrics from wearable sensors hasn’t been yet tackled.

To address this issue, the research was focused on Pedestrian Dead Reckoning, using Inertial Measurement Units.
Dead reckoning is a technique used for estimating an object’s position or its trajectory considering their current speed and direction.
It is mainly used by marine or air navigation, but recently it has been applied to pedestrians.

Pedestrian Dead Reckoning is based on the usage of accelerometers to detect a step and estimate step-length, and gyroscopes to compute changes in direction. Magnetometers can be also used to determine the orientation according to Earth’s North Pole, but this method might fail due to magnetic interferences, as described in Section 2.2.2.3.

There are several proposed algorithms to calculate a subject position using IMUs. Carl Fischer, Poorna Sukumar and Mike Hazas propose an approach to implement a tracker using a Kalman Filter and correcting the velocity using zero-velocity updates.
Sebastian Madgwick presents a novel orientation algorithm, that claims to be more accurate than the Kalman based algorithm. Both approaches use a shoe-mounted sensor.

When this work started, the devices were only sending data of the accelerometer sensor. In order to retrieve data from the gyroscope and magnetometer, a special command had to be sent to the device to activate the sensors.

I started to implement the Madgwick algorithm, using the code available publicly in , gathering data from the available sensors. The results were poor, as the path calculated by the algorithm was off by thousands of meters. This could be because of the high frame rate needed by the algorithm (512 HZ = 2ms). At this rate, our sensor skips a high number of samples, loosing essential parts of the data set.

The following step was to try the Tutorial approach. Better results were achieved using this algorithm, but there were still some tweaks to make, and an error associated with the position of the sensor on the foot and its displacement while walking.
Initially the sensor was placed below the shoe laces, but we found that the sensor moved while walking, and the position of the sensor was tilted, which introduced a deviation in the results.

Afterwards, the sensor was attached to the foot using 3 small rubber band. While walking, the sensor moved to front and to the right of the right foot. This caused a deviation in the route traced by the algorithm, as shown in the figure.