%!TEX root = ../elaboration.tex
\chapter{Conclusions and Future Work}
\label{cha:conclusions}
\section{Conclusions}
This work aimed to cover a gap in the area of basketball player tracking systems, with a special focus on amateur and college teams, which usually don't have the financial means to acquire current commercial technologies, that can range from tens to hundreds of thousands of euros.

As stated in Section~\ref{sec:objectives}, this dissertation proposes a system that can collect data from multiple sensors placed in the body of the players of a team, and calculate game metrics from this data. This processed data should be stored for future analysis and statistics, and it should also be presented to coaches, in an application. In this application, the coaches are able to view detailed information about their players and team performance, in real time. 

Addressing the question of collecting data being transmitted at high rate from multiple sensors at the same time, processing and storing it, with varying devices and distances, the architecture developed showed good results, showing low losses in the data transmitted. The architecture developed  is also easily scalable, allowing coaches to choice how many sensors they want in each player, without great expenses. 

Regarding individual player metrics, four game metrics identification algorithms where developed. These four where chosen because all except the dribble identification had a common ground with other sports, and could be used in other similar systems, applied to different sports. 
Two of the developed metrics, step detection and dribble detection, performed very  well, with results over 95\%.  

In Section~\ref{sec:objectives}, it is stated that the system should also be capable to provide an aggregated view of the team performance, like attacking and defending strategies, and to show the position of the players during the game in the form of heatmaps. However, these team metrics relied heavily on the positional tracking. Due to the poor results showed in Subsection~\ref{subsec:positionresults}, these metrics where not developed.

The frontend application developed served as a support to test the performance of the developed game metrics identification, and was developed according to the coach's perspective, providing operations to control and analyse their players performance. There are missing features in the application, that are available in the \gls{API}, like the ability to edit and delete games, players and teams.

Looking at the big picture, most goals where achieved, even though some where only partially completed, and the ability to track team sports players and collect game metrics can be achieved using low cost \gls{IMU} sensors and processing devices, providing amateur and college teams with analytic tools that where usually out of budget and only available for professional teams.


\section{Future Work}
This work left some loose ends, due to time constraints, and further improvements can be developed in future work.

First of all, the metrics algorithms where all tested in isolation, and where tested by me and by co-workers in a controlled environment, performing scripted movements. A real-world test should be made, ideally using a team practice or a game between two teams to collect data from the players, and evaluate if the performance of the algorithms keeps up with the rapid movements of a basketball game.

Regarding the metrics identification algorithms, the algorithms with the worst results should be improved, and more advanced classification techniques could be used, like machine learning methods, as seen in the related works, in Section~\ref{sec:relatedWork}. However, this would mean that a large amount of data should be collected to provide to the machine learning classification methods.

Finally, the system could store more statistical information about the players, the teams and the games. For example, data about the player's age, height, weight and former teams, data about the team ranking in the league, the city it belongs to and the team's current, and info about the game's location, lineups and box score. This additional info would serve to provide to the fans a version of the frontend application with detailed stats about the players, the teams and the games.