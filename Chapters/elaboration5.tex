%!TEX root = ../elaboration.tex
\chapter{Results}
\label{cha:results}

\section{Data Availability and Quality}
\label{sec:resultsData}
The evaluation of the data will be separated in two parts: the availability and the quality of the data.
The availability will measure if all the raw data sent by the \gls{IMU} Sensors is received by the Raspberry Pi's, and what factors affect it.
The quality will measure the degree of truthfulness of the raw data measured by the \gls{IMU} Sensors.

\subsection{Data Availability}
To measure the availability of data, it is retrieved from the \gls{IMU} Sensors, at 50Hz, from the accelerometer and gyroscope. No computations are made on the data. This means that the Raspberry Pi should receive 50 samples of Raw Data in 1 second. Measuring how many samples are missing, we can calculate the percentage of lost samples.

This measurements will be made in an open space, and 3 factors will change:
\begin{enumerate}
    \item Distance
    \item Number of \gls{IMU} Sensors
    \item Raspberry Pi version (3B+ vs 4)
\end{enumerate}

\begin{table*}[ht]
%\vspace{-0.2cm}
\caption{Percentage of data loss (Raspberry Pi 3b+).}\label{tab:datalossrPi3} \centering
    \begin{tabular}{lllllllll}
              & 0,6m & 1,2m & 3m    & 6m    & 9m    & 12m   & 15m   & 18m   \\ \cline{2-9}
    1 Device  & 4,27 & 4,39 & 11,25 & 18,22 & 28,04 & 15,43 & 28,16 & 19,79 \\
    3 Devices & 5,65 & 7,68 & 12,50 & 18,79 & 21,76 & 16,60 & 39,50 & 26,78 \\
    6 Devices & 6,52 & 6,05 & 15,53 & 16,31 & 21,64 & 19,77 & 32,90 & 21,78 \\
    9 Devices & 6,17 & 7,63 & 12,80 & 19,98 & 28,65 & 36,62 & 40,37 & 32,96 \\
    \end{tabular}
\end{table*}

\begin{figure}
    \centering
    \includegraphics[width=\textwidth]{PacketLoss(RaspberryPi3).pdf}
    \caption{Percentage of data loss (Raspberry Pi 3b+)}
    \label{fig:datalossrPi3}
\end{figure}

\subsection{Data Quality}
\lipsum[1-2]

\section{Metrics Performance}
\label{sec:resultsMetrics}
\subsection{Trajectories}
\lipsum[1-2]

\subsection{Steps}
\lipsum[1-2]

\subsection{Dribbles}
\lipsum[1-2]

\subsection{Jumps}
\lipsum[1-2]
% \begin{enumerate}
%     \item Disponibilidade dos dados
%     \begin{enumerate}
%         \item de acordo com a Distância
%         \item de acordo com o nº de dispositivos
%         \item de acordo com a versão do rPi
%     \end{enumerate}
%     \item Qualidade dos dados
%     \begin{enumerate}
%         \item registos errados (outliers, gravdiade errada)
%         \item Ruido associado
%     \end{enumerate}
%     \item Fiabilidade das métricas
%     \begin{enumerate}
%         \item Dribbles
%         \item Saltos - Os saltos têm de ter mais de 30cm, porque o algoritmo apenas deteta estes, se não confundia com corrida
%         \item Passos/Distância
%     \end{enumerate}
% \end{enumerate}