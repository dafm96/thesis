%!TEX root = ../template.tex
%%%%%%%%%%%%%%%%%%%%%%%%%%%%%%%%%%%%%%%%%%%%%%%%%%%%%%%%%%%%%%%%%%%%
%% abstrac-pt.tex
%% NOVA thesis document file
%%
%% Abstract in Portuguese
%%%%%%%%%%%%%%%%%%%%%%%%%%%%%%%%%%%%%%%%%%%%%%%%%%%%%%%%%%%%%%%%%%%%
Desportos colectivos tais como o basquetebol, futebol ou basebol usam dados para melhorar o desempenho em jogo, o planeamento de treinos e a recuperação de lesões de jogadores e da equipa. O aperfeiçoamento de técnicas como \textit{data mining} e \textit{machine learning} têm levado à pesquisa de formas de recolher mais e melhores dados dos jogadores.

Hoje em dia há várias abordagens diferentes: usando sistemas de captura de video ou utilizando sensores. Esta última abordagem pode ser bastante precisa no posicionamento em exteriores, utilizando sistemas como o GPS, ou em interiores, utilizando sistemas de posicionamento como Ultra-Wideband ou RFID. No entanto, estes sistemas não efetuam o reconhecimento de métricas de jogo como acelerações ou saltos, ou passes e lançamentos no basquetebol 

Este trabalho propõe uma solução que visa medir em tempo real o posicionamento de jogadores e da equipa no campo, assim como recolher métricas de jogo, utilizando sensores de inércia. Deste modo, pretende-se fornecer aos utilizadores (e.g.\ jogadores, treinadores) uma aplicação com informações úteis e detalhadas sobre o desempenho dos jogadores individualmente, e da sua equipa como um todo.
