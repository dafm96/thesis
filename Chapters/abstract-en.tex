%!TEX root = ../template.tex
%%%%%%%%%%%%%%%%%%%%%%%%%%%%%%%%%%%%%%%%%%%%%%%%%%%%%%%%%%%%%%%%%%%%
%% abstrac-en.tex
%% NOVA thesis document file
%%
%% Abstract in English
%%%%%%%%%%%%%%%%%%%%%%%%%%%%%%%%%%%%%%%%%%%%%%%%%%%%%%%%%%%%%%%%%%%%
Team Sports like basketball, football and baseball rely on data insights to improve player and team performance, practice scheduling and injury recovery. Recent improvements in data mining and machine learning techniques have encouraged the research of better ways to collect player data.

There are different approaches currently available, using video system or wearable sensors. While these systems can be very precise in tracking position outdoors using GPS or indoors using Ultra-Wide band or RFID positioning, they fall short in the analysis of game related metrics, like accelerations and jumps or passes and shoots in the case of Basketball.

These metrics can be tracked using wearable Inertial Measurement Units, and the solution proposed by this dissertation is a system using Inertial sensors that can track both position and metrics of players and teams in real-time, giving coaches insightful information in order to improve the performance of the individual player and the whole team.
