%!TEX root = ../elaboration.tex
\chapter{System Implementation}
\label{cha:implementation}
\noindent -----//-----
\begin{enumerate}
    \item Raspberry Pi
    \begin{enumerate}
        \item quais os algoritmos
        \begin{enumerate}
            \item Dribbles
            \item Saltos
            \item Passos/Distância/Posição
            \item Tempo parado/andar/correr
        \end{enumerate}
        \item operações sensores
        \item como se controlam os sensores
        \item como são enviados os dados (mqtt)
        \item feito em node
    \end{enumerate}
    \item Server
    \begin{enumerate}
        \item REST API feita em Node
        \item DB em mySQL, explicar modelo
        \item como troca mensagens com o rPi
    \end{enumerate}
    \item Cliente
    \begin{enumerate}
        \item Feito em React
        \item Permite gerir equipas, jogadores e jogos
        \item métricas por jogo, vários sensores por jogador
    \end{enumerate}
\end{enumerate}aaa
-----//----- 

\noindent The description of the System Implementation according to the proposed architecture will be separated in the three segments: Edge, Server and Client.

\section{Edge}

The edge segment, composed by one or more IMU Sensors and one or more Raspberry Pi, as described in section~\ref{sec:edge}, is the scalable part of the architecture. 

The number of IMU sensors used can vary in each player by the number of metrics being measured (different metrics may require the use of more sensors) and the number of players being tracked.

In order to accommodate the physical limitation of Bluetooth connections, 
%TODO support this affirmation
one Raspberry Pi may not be sufficient to handle all the IMU Sensors. 
It is necessary to ensure that the communication flows from the Server to the IMU Sensors being just a mean of distributing the correct data to and from the multiple sensors.


\subsection{IMU Sensor Control and Communication}
\noindent The IMU Sensors used in the implementation were developed by AITEX.
%REFERENCE gls AITEX
They communicate through Bluetooth Low Energy and have 3 sensors built-in: 3-axis accelerometer, 3-axis gyroscope and 3-axis magnetometer.

The following operations are supported by IMU Sensors:
\begin{itemize}
    \item State control
    \begin{itemize}
        \item Start Raw Data Collection
        \item Stop Raw Data Collection
        \item Shutdown
    \end{itemize}
    \item Sample Rate
    \item MPU Configuration
\end{itemize}

The IMU Sensors must pair with a Raspberry Pi, which is searching for the known sensors. After being paired, the operations can be communicated to the IMU Sensors through Bluetooth Notifications. 

Before starting the data collection, it is necessary to set the sample rate to 50 Hz (necessary for the metrics algorithms), and to activate the gyroscope and the magnetometer sensor by changing the MPU Configuration. By default only the accelerometer sensor is active.

To start collecting data from an IMU Sensor, it is needed to provide information like the MAC Address of the sensor, an identification of the game being played and the player wearing the sensor (because the sensors can be changed from game to game between the players). It is also needed to send the position of the sensor in the player's body, to calculate the metrics, as each algorithm uses data collected from a different part of the body.
When the metrics are calculated, they are sent to the server via MQTT, to be stored there.
%TODO citar mqtt?
%TODO gls MQTT
The Raspberry Pi's don't store data, they only server as a vehicle of instructions and data between the server and the sensors. 

A single Raspberry Pi is then responsible by maintaining the connection with several IMU Sensors; communicate instructions; receive accelerometer, gyroscope and magnetometer data, and stores it temporarily; depending on the position of the sensor, calculate the according metrics algorithms with the received data; send the calculated metrics to the Server.


\subsection{Metrics Algorithms}
\subsubsection{Steps}

\subsubsection{Dribbles}

\subsubsection{Jumps}

\subsubsection{Trajectories}

\section{Server}
The server is built as an Rest API, developed with Node.js, and taking advantage of the Express framework that enables a fast development of Web Applications and API's. %REFERENCE nodejs, rest, api, express

To manage peripherals (IMU Sensors), Players, Teams, and Games, there are API endpoints for each one of this resources, and all the data is saved in a MySQL Database.

\subsection{API Description}
The Developed API exposes endpoints to control the different resources available in the system. Those resources are:
\begin{itemize}
    \item Peripherals (IMU Sensors)
    \item Players
    \item Teams
    \item Games
\end{itemize}

\subsubsection{Peripherals}

\subsubsection{Players}

\subsubsection{Teams}

\subsubsection{Games}



\subsection{Database Model}

The database, developed in MySQL, was designed in order to store all the data from the main resources, its relations and meaningful data in those relations.

%REFERENCE reference figures

The main resources that need are represented in the following tables: Peripherals; Players; Teams; Games; Metrics. 

Each entry in the Peripheral table represents a IMU Sensor, and it's stored the peripheral MAC Address, and its attributed number for easier identification.

In the Players table, each player has stored its name, and a identifier of the team they belong to. %could store age, height, weight, position

The Team table only stores the name of the team. %could store city, year of foundation, score in the league, coach....

To store information about a game, the Game table has the date of the game, and a reference for both teams attending the game. %could store where it is, the score of the game, the referee....

For the metrics, each player has a set of metrics per game. 

One of the features of the system is to be able to handle multiple peripherals per player, in different body positions, that can be changed between games. To store this information, two tables are needed: one that stores the relation between the player and the game, and another that relates that relation with a peripheral, and stores 

\section{Client}
The client application was developed in React, with the goal of controlling the IMU Sensors, and manage teams, players and games. It should also display Player and Team game-related metrics.