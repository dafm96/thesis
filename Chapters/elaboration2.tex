%!TEX root = ../elaboration.tex

\chapter{State of the art analysis}
\label{cha:relatedWork}
\epigraph{Many studies carried in the past have focused on subjects related with this work, such as collecting human movement data, human activity recognition and the measuring of game metrics in sports.
As defined in Section~\ref{sec:objectives}, this work will be about collecting data from basketball players, to analyse it and calculate game metrics, that can be useful to the coaches and to the fans.
The following will discuss the related works that served as a starting point to this dissertation and will present some current commercial solutions.}

\section{Related work}

Several authors propose means of transforming data from \gls{IMU} Sensor into meaningful metrics. This section divides the related works in two: first, the related works regarding human activity recognition such as daily activities like walking, climbing stairs and pedestrian tracking will be analysed; then, related works that identify particular basketball activities such as dribbles, passes and shoots are discussed.

\subsection{Human Activity Recognition}
In the area of general human activity recognition, Kerem Altun and Billur Barshan~\cite{Altun2010} propose a method to identify several human activities like sitting, standing, walking, climbing stairs and many other, using data from 5 sensors placed in both legs, both arms and in the chest. 
After extracting features and reducing the data, different classification techniques are employed to classify 19 activities. They conclude that Bayesian Decision Making is the best method to identify this type of activities, followed by k-Nearest Neighbor, both performing over 95\%.

Charissa Ann Ronao and Sung-Bae Cho~\cite{Ronao2016} approach the problem of the human activity recognition by proposing a deep convolutional neural network. With data collected from the accelerometer and gyroscope of a smartphone placed in the pocket, they can identify activities walking, sitting and standing, and walking upstairs and downstairs, with a precision of over 94\%.

Hassan \textit{et al.}~\cite{Hassan2018} propose a similar method, also using data from the smartphone's accelerometer and gyroscope, but propose a different classifier. Using a Deep Belief Network, the authors compare their method with other commonly used classification methods, and present better results, with a 95.85\% accuracy.

Regarding pedestrian tracking, Wonho Kang and Youngnam Han~\cite{Kang2015} propose an indoor pedestrian location system using off-the-shelf smartphones. With the embedded accelerometer, gyroscope and magnetometer sensors, they perform step detection, estimate the step length and heading direction to trace the path. The results show a satisfiable tracing of the trajectory of the subject with respect to the ground-truth reference. 

Carl Fischer, Poorna Sukumar and Mike Hazas~\cite{Fischer2013} propose a tutorial to implement a pedestrian tracker, using a inertial sensor placed in the foot. By employing techniques like Kalman Filtering to estimate the system error from the measurements of the accelerometer and gyroscope sensors, and Zero-Velocity measurements, they can correct the velocity and position estimates, resulting in a close to ground-truth dead reckoning measurements.

Sebastian Madgwick~\cite{Madgwick} presents a novel orientation algorithm, that claims to be more accurate than the Kalman based algorithm.
Both approaches use a shoe-mounted sensor.

\subsection{Basketball Activity Recognition}
N. F. Ghazali \textit{et al.}~\cite{Ghazali2018} compare different machine learning techniques to identify common sport activities like being stationary, walking, running, sprinting and jumping, with data collected from an inertial sensor strapped to the chest. The raw data (accelerometer and gyroscope) was labeled, and features were extracted from this data, filtering the relevant information. Then, 5 different types of classification techniques were employed, and compared between each other. The best classifier achieved an accuracy of 91\%, successfully classifying common sport activities.

Le Nguyen Ngu Nguyen \textit{et al.}~\cite{Nguyen2015} use a multi-sensor system composed by 5 sensors placed in both feet, both legs and in the back. With data collected from the accelerometers, they aim to identify general activities like walking, jogging and running, but also basketball-specific activities like jumpshot, layupshot and pivot. The raw data is firstly preprocessed and segmented, and features are extracted from this data. Then, based on the z-axis acceleration, they separate standing and moving activities. Using a Support-Vector-Machine-based classifier, the moving activities are classified. Although the classifications aren't very accurate, they conclude that more information from the sensors should be used, like the data from the gyroscopes, and that to correctly classify similar activities like dribbling (\textit{i.e.} running with the ball) and running, more sensors should be placed in different body parts (\textit{e.g} on the wrists), to differentiate these activities.

Xiangyi Meng \textit{et al.}~\cite{Meng2018} used the same classification technique, but collected Raw Data from accelerometer and gyroscope sensors, and the sensors were placed in the wrist. The authors only aimed to identify 3 basketball activities: shoots, passes and dribbles. Even though a result of 96\% is achieved for dribbling classification, shoots are easily confused with passes. 

Ruijie Ma \textit{et al.}~\cite{Ma2018} uses a wrist mounted \gls{IMU} Sensor in order to identify the posture of basketball player, and recognize nine other basketball movements, such as stand, walk, run, jump, in-situ dribble, dribble while walking, dribble while running, set shot, and jump shot. Using a neural network algorithm to identify this movements, an accuracy of 98.9\% is achieved, but only a subject provided the data for the test and validation phases.

Alexander Holzemann \textit{et al.}~\cite{Holzemann2018} has a similar approach, with wrist-worn \gls{IMU} Sensors, identifiyng activities such as dribbling, shooting, blocking, or passing. Using classifiers, a performance of 84\% is achieved with k-Nearest Neighbor classifier and 88\% with Random Forest classifier, with actions like jump shots performing better than dribbles.

%TODO conclusões destes papers

\section{Commercial Solutions}
The most used way of track Basketball players is using video tracking systems. Currently, Second Spectrum is the official optical tracking provider for the \gls{NBA}~\cite{NBAstuffer}. This system uses cameras located around the court to collect 3d spatial data of player and ball location and movements, and play-tracking data like speed, distance, shoots, passes, drives and more~\cite{AWSMe}.

Then, using machine learning algorithms, players and coaches can get detailed reports and insights about player performance. This approach can also be profitable for leagues and media, providing data to interactive applications or using augmented reality video to improve the fan experience.

SportVU is an Optical Tracking system, used by Stats Perform~\cite{StatsPerform}, that with cameras placed around the court and the use of artificial intelligence techniques can gather innovative statistics based on player speed, distance traveled, separation, ball possession and more. They combine the tracking of the player and the ball with historical databases to provide in-depth insights for team and player performances. This optical system is available for basketball and football, and besides providing useful information for team management, this data is also used for fan media, and for betting and fantasy content.

Many other sports use video tracking systems in their games, and they can give very accurate measurements about the players location and team tactics, and even track small movements like shoots and passes in Basketball. However, these systems require high processing power, and are very expensive, being out of reach for minor and amateur teams. Video systems are also attached to the field, and in order to track a team in a different location, the system needs to be installed there. Another drawback is that some video systems don’t work in real-time. They need an operator to tag events, and the processed data is deferred~\cite{VanderKruk2018}.

Other way to track players is using a sensor based approach. Multiple commercial products are used by both amateur and professional teams.

Kinexon~\cite{Kinexon20} developed a player tracking system based on \gls{UWB}. It is comprised of wearable player sensors, that are equipped with 9-axis accelerometer, gyroscope and magnetometer, and an array of base-sensors placed around the court. It can deliver precise 3D-position data, by tracking the players using \gls{RF} technology,  and detects metrics like jumps, sprints, impact, changes of direction, accelerations and deceleration, among many others. It also works both indoors and outdoors.~\cite{WearableTechnologies}.

ShotTracker~\cite{ShotTracker} is tracking system that is made of 3 components: ShotTracker-enabled ball, player sensors and anchors around the court. The Anchors are what locates the players and the ball, and can infer metrics like distances, number of passes, shots and shot types, and have a structured summary of the results of a game. All the data that is collected is analysed and displayed in an application in real-time. The data presented in this application has three targets: the coaches, to improve their team's performance; the players, to analyse and enhance their practice; and the fans, giving them access to advanced team stats and detailed box scores, and even Augmented Reality features to interact with the live game.

STATSports~\cite{STATSports} is a system developed for football player tracking. The players use a pod worn in the back, that is equipped with an accelerometer, a gyroscope, a magnetometer and that collects positional tracking using a GPS device. This device has the ability to calculate internally over 50 metrics in real-time (\textit{e.g.} total distance covered, number of accelerations or decelerations and heart rate), that are then transmitted over a \gls{UWB} network. A cloud-based platform aggregates all the data, and presents available advanced metrics dashboards and individual/group data for players and coaches to consult. It is used widely by several professional football teams.

All the sensor-based systems presented above claim  to be very precise in the metrics they measure, and many of them are used by professional teams in their competitions. However, they all rely on a infrastructure placed in the court, or on external tracking solutions (like \gls{GPS} in the case of STATSports). For Basketball, the use of \gls{GPS} is out of question, as most of the Basketball courts used by amateur teams are indoors. The need of a infrastructure is also not ideal for amateur teams, because it increases the costs and hinders the ability to play in different fields, which are two important factors for amateur teams.